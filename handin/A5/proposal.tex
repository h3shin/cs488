\documentclass {article}
\usepackage{fullpage}

\begin{document}

\noindent{\Large \bf Final Project:}
\begin{description}
\item[Purpose]:\\
	To tie together three totally unrelated rendering issues.

\item[Statement]:\\
	For Ray Tracers: Paragraph describing interesting scene to be
		rendered and what features are needed to achieve
		this scene.

	Paragraph: What it's about.

	Paragraph: What to do.

	Paragraph: Why it is interesting and challenging.

	Paragraph: What I will learn

\item[Technical Outline]:\\
    Basically, your objectives in your objective list should be fairly
    short statements of the objective; you should provide additional
    details about your objectives in this section to clarify what you
    plan to do.

     Further, survey the important data structures and algorithms that
     will be necessary to achieve the goals, and (for ray tracing
     projects) lists the new commands
     that will need to be added to the input language.

     To  get  bold face: {\bf bold face words}.  To get italics: {\it italic
     face words}.  To  get typewriter font: {\tt typed words}.  To get
     larger  words:  {\large large  words}.   To  get smaller words: 
     {\small small words}.  

\item[Bibliography]:\\
     Articles  and/or  books  with  important  information on the
     topics of the project.

\end{description}
\newpage


\noindent{\Large\bf Objectives:}

{\hfill{\bf Full UserID:\rule{2in}{.1mm}}\hfill{\bf Student ID:\rule{2in}{.1mm}}\hfill}

\begin{description}
     \item[\_\_\_]  Objective one.

     \item[\_\_\_]  Objective two.

     \item[\_\_\_]  And so on.
\end{description}
\end{document}
